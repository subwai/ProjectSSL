\section{SSL/ TLS}
SSL är ett väldigt väl använt protokoll som man finner på nätet när en säker anslutning krävs. Med SSL förhindrar man tjuvlyssning på nätverket, vilket är en viktig faktor som gör protokollet säkert. 

Protokollet befinner sig mellan de två översta lagren i TCP/IP modellen, transportlagret och applikationslagret.
Eftersom protkollet används för säkra anslutningar på nätet faller det i sin natur att SSL används mestadels i Hypertext Transfer Protocol (HTTP) [första referensen: http://docs.oracle.com/javase/7/docs/technotes/guides/security/jsse/JSSERefGuide.html]. 

Många andra applikationer kan dra nytta av SSLs säkra anslutningar, några av dem är Telnet och FTP. 
SSL protokollet har två lager av protokoll vilket har olika funktioner [inludera en bild på det].

\subsection{Record Layer}
I Record Layer ser protokollet till att meddelandena får integritetsskydd. För att göra detta så använder sig SSL av symmetriska nycklar och digitala signaturer. Det som utförs är att det beräknas en MAC för packetet som ska skickas med hjälp av HMAC funktionen. HMAC funktionen är till för att kunna kontrollera datan om den är säker över en opålitlig länk. Parterna som använder sig av HMAC delar på en nyckel för att kunna utföra samma säkerthetskontroll. HMACen används med en hashfunktion som till exempel MD5, och därav ett h framför MAC. MACen kommer tillsammans med datan krypteras och vara datapacketet av Record protokollet. Utöver detta kommer det finnas en header som innehåller information om hur en 'handshake' går till, vilken version av SSL/ TLS som används samt hur mycket data det finns i paketet.

\subsection{Upper Layer Protocols}
I Upper Layer Protocols är det flera steg som utförs. Här kommer protokollet tilldela nycklar och autentisera klienter respektive servrar med hjälp av assymetriska nycklar. När protokollet gör detta betyder det att protokollet startar sin 'handshake' metod. Utöver själva autentiseringen av server till klient så tillhandager 'handshake' också:


\begin{enumerate}
\item{Fastställa vilka algoritmer som ska användas}
\item{Förhandla om vilka nycklar krypteringen skall använda samt vilken MAC som skall användas}
\item{Autentisera klienten till servern, vilket inte är nödvändigt}
\end{enumerate}

Eftersom denna 'handshake' kan bestå av delar eller hela meddelandecyklar, beskriver vi hur vårt system använder sig av dessa meddelanden i vår 'handshake' metod.

\subsection{Handshake}