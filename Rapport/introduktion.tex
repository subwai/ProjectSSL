\section{Introduktion}

När man som individ ska köpa något i en affär eller hämta ut pengar från banken så brukar det krävas att man visar legitimation. Eftersom man fysiskt befinner sig hos personen som tillhandahåller det du vill köpa/ hämta är en legitimation det enda du behöver som säkerthet. För att denna legitimation ska vara pålitlig så ska den vara utförd av statligt organ. Att autentisera sig som person på detta sätt är väldigt svårt att imitera med förfalskade legitimationer. Men om du tänker föra över pengar eller köpa något över Internet så är säketheten annorlunda. Att visa en legitimation går inte eftersom du aldrig kommer vara närvarande fysiskt när överföringen/ köpet utförs. Här finns det säkerhetsrisker både hos personen som tillhandahåller hemsidan samt hos dig som person. Det man då borde fråga sig är om hemsidan man besöker är rättmätig och det hemsidan kommer fråga sig är om det verkligen är du som försöker utföra köpet/ överföringen från ditt personliga konto.

För att kunna bedriva transaktioner över nätet är det extremt viktigt för aktörer att förse sina tjänster med säkerhetssystem som kräver olika former av autentisering. Vi kommer att studera protokollet SSL som tillför säkerhet mellan en klient och server. Detta protokoll tillför även kryptering av datan som skickas över länken, vilket gör att svårigheten att knäcka överföringen ökar. Protokollet kan stå emot ett antal olika attacker vilka ska säkerhetsutvärderas och sammanfattas. 
